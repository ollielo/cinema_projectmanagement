\documentclass{article}
\usepackage{graphicx}

\begin{document}

\title{Cinema Use Cases for Spec D}
\author{David Rogers}

\maketitle

\section{Introduction}
We are proposing changes to the Cinema specification that expand the
capabilities of Spec D to relax some of the current constraints and expand the
types of artifacts that can be written to the database. In particular, we are
moving from mapping a set of parameters to a set of artifacts (files), to
a collection of several mappings of parameters to several sets of artifacts
(files). These use cases will be explained in the examples below.

In particular, this new design will support several types of mappings of
parameters to artifacts, all of which are possible in the spec, and must be
managed by the producer and consumer of the database. This is explained in
section \ref{s:mappings}.

\section{Mappings of Parameters to Artifacts} \label{s:mappings}
This is the section on mappings.

\section{Use Cases}
Two different artifacts, each at the same camera position and time:
\begin{center}
\begin{tabular}{l|l|l|l|l}
phi & theta & time & FILE & FILE \\
\hline
0.0 & 0.0 & 1.0 & 00-00-10-00.png & \\
1.0 & 0.0 & 1.0 & 10-00-10-00.png & \\
2.0 & 0.0 & 1.0 & 20-00-10-00.png & \\
3.0 & 0.0 & 1.0 & 20-00-10-00.png & \\
0.0 & 0.0 & 1.0 & & 00-00-10-01.png \\
1.0 & 0.0 & 1.0 & & 10-00-10-01.png \\
2.0 & 0.0 & 1.0 & & 20-00-10-01.png \\
3.0 & 0.0 & 1.0 & & 20-00-10-01.png \\
\end{tabular}
\end{center}

A single artifact consisting of two images:
\begin{center}
\begin{tabular}{l|l|l|l|l}
phi & theta & time & FILE & FILE \\
\hline
0.0 & 0.0 & 1.0 & 00-00-10-00.png & 00-00-10-01.png\\
1.0 & 0.0 & 1.0 & 10-00-10-00.png & 10-00-10-01.png\\
2.0 & 0.0 & 1.0 & 20-00-10-00.png & 20-00-10-01.png\\
3.0 & 0.0 & 1.0 & 20-00-10-00.png & 30-00-10-01.png\\
\end{tabular}
\end{center}

A single artifact consisting of two images at a specific time, and another
consisting of a single grid file for the data. The timesteps for which the grid
is valid is not provided in the Cinema database:
\begin{center}
\begin{tabular}{l|l|l|l|l|l}
phi & theta & time & FILE & FILE & FILE\\
\hline
0.0 & 0.0 & 1.0 & 00-00-10-00.png & 00-00-10-01.png & \\
1.0 & 0.0 & 1.0 & 10-00-10-00.png & 10-00-10-01.png & \\
2.0 & 0.0 & 1.0 & 20-00-10-00.png & 20-00-10-01.png & \\
3.0 & 0.0 & 1.0 & 20-00-10-00.png & 30-00-10-01.png & \\
 &  &  & & & out.grid\\
\end{tabular}
\end{center}

%\begin{figure}
%    \centering
%    \includegraphics[width=3.0in]{myfigure}
%    \caption{Simulation Results}
%    \label{simulationfigure}
%\end{figure}

\end{document}
